%\addchapter{References}
\addchapter{参考文献}
\begin{thebibliography}{9}% 文献数が10未満の時 {9},10~99の時 {99}
\bibitem{saito}
齋藤 正彦:
線型代数入門,
東京大学出版会,1966

\bibitem{takagi}
高木貞治:
初等整数論講義(第2版),
共立出版,
1988(昭和53)

\bibitem{knuth}
D.E.~Knuth:
{\it The Art of Computer Programming},
2nd Ed.,
vol.1 Fundamental Algorithm,
Addison-Wesley,
1973

\bibitem{wozencraft} 
J.M. Wozencraft, 
I.M. Jacobs,
{\it Princeples of Communication Engineering},
John Wiley \& Sons, Inc., 1965

\bibitem{proakis} 
J.G. Proakis, 
{\it Digital Communications},
McGraw-Hill, 1995
(邦訳:ディジタルコミュニケーション,科学技術出版社,1999)


%%%%%%%%%%%%%%%%%%%%%%%%%%%%%%%%%%%%%%%%%%%%%%%%%%%%%%%%%%%%%%%%%%%%%%%
\bibitem{BCH-1}
A.~Hocquenghem,
``Codes correcteurs d'erreures,''
{\it Chiffres}, vol.2, pp.147--156, 1959

\bibitem{BCH-2}
R.C.~Bose and D.K.~Ray-Chaudhuri,
``On a Class of Error Correcting Binary Group Codes,''
{\it Information and Control}, vol.3, pp.68--79, 1960

\bibitem{dfr}
G. L. Feng and T. R. N. Rao,
``Decoding Algebraic-Geometric Codes
up to the Designed Minimum Distance,''
{\it IEEE Trans. Inf. Theory},
vol.IT-39, pp.37--45, 1993

\bibitem{majime}
真面目 楽太郎,
``楽に卒業する方法について,''
電子情報通信学会論文誌,
vol.J101-Z, no.13, pp.5398--6421, 2022

\end{thebibliography}
