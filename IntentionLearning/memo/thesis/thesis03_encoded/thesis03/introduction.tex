%\chapter{Introduction}
\chapter{序論}

\section{背景}

近年のロボット技術の発展により、人間の生活環境で活躍する汎用ロボットの実現に向けた様々な研究が行われている。汎用ロボットとは、工場などで作業を行うようなあらかじめプログラムされた動作を繰り返すものではなく、環境との相互作用によって適切な行動を学習、選択して実行できる能力を持ったロボットのことである。人間の生活環境で活躍する汎用ロボットには、人間とのインタラクションを通じて動作の方法や内容を学習する能力が必要不可欠である。
人間とのインタラクションを通じて学習を行う手法の研究として、人間の教示動作を観測しそれを模倣することによって、ロボットがその動作を再現するにはどのようにすればいいかを学習する模倣学習があるが、
人間の動作を模倣するだけではその動作を理解したとは言えない。
例えばロボットに、コップを運ぶという動作を教示して再現させることを考える。ある時点で食器棚に置かれているコップを、その時点でテーブルの一角に座る人の前に動かす動作を見せた場合、単純に教示動作からコップの最終位置を把握したとしても、それはそのテーブルの一角という場所に動かすことが重要なのか、その座っている人の前に動かすことが重要なのかが理解できていなければ、人の座る位置やテーブルの位置などの環境が変わった場合に教示者の意図に正しく沿った動作が再現できない。このような問題を解決するためには、特定の初期状態から人間がどのように動作したかという明示的な情報から、人間がその動作を行う際に環境中のどの情報を意識しているかという暗黙的な情報を推測し、動作における観点を人間とロボット間で共有する必要がある。そのような背景から、本研究では教示者の動作から着目している観点を推測することで教示者の動作意図を把握し、意図を汲み取った動作を再現できるような、気がきくロボットの実現を目指す。

\section{関連研究について}

人間とロボットのインタラクションの分野において、
人間の教示動作によるロボットの模倣学習に関する様々な研究が行われている。([0]-[0])模倣学習とはロボットに対し、タスクや動作をあらかじめプログラムするのではなく、ロボットの腕を直接操作したり、人間の動作を視覚的に与えるなどで動作を教示し、その挙動を模倣することでタスクや動作を学習する手法を研究する分野である。その際に教示者とロボット間のキネマティクスなどの制約の違いを克服する機構を持たせることで、ロボットへの動作教示をより自然、適切に行うことを可能に。。。
〇〇は、人の舞踊動作を見せることで、四肢の長さや関節角などを考慮した見真似学習の方法について述べている。(※この引用いらないかも)
〇〇は人教示したの舞踊動作から四肢の長さや関節角などを考慮した見真似学習を行い、同時に安定した姿勢を保ち直立を維持した状態でロボットに舞踊動作を模倣させる研究を行った。このようにロボットに対し、教示動作と類似した動作を再現させる手法としての模倣学習は。。。
Schaal[2]は、DMPと呼ばれる運動モデルのパラメータを教示動作から学習することで、ドラムをたたくなどの周期運動や、ラケットでボールを打つなどの到達運動の再現を行っている。このように、目標位置が既知で教示者とロボットの間で共通に認識されているという前提で、その目標位置までどのように動作するかを扱う研究は多く存在するが、動作教示の煩わしさを軽減し、人とロボットのインタラクションを円滑にするためには、教示者の意図する目標位置は明示的に与えるより、教示動作からロボットが推測し、教示者は教示動作を見せるだけでよくなることが望ましい。そのため、目標位置はロボット自身が教示動作から推定する機構を持たなければならないが、目標位置の決定方法は動作によって異なる。例えばピックアンドプレースにおいて、物体1を右に動かすという動作は物体1の位置が最終位置の決定にかかわるが、物体1を物体2に近づけるという動作は物体1と物体2の位置関係が最終位置の決定に必要になる。これには、目標位置決定の基準となる参照点と、その参照点に対してどのような位置、状態かの2つを推定する必要がある。
杉浦ら[3]は動詞の持つ動作概念を人とロボット間で共有することを目標にした研究を行った。その中でピックアンドプレースの教示に関して、各オブジェクトの位置を参照点とし、複数の座標系に変換した動作軌跡を生成する隠れマルコフモデルを学習して最尤推定によって参照点を決定し、参照点を考慮した動作概念の獲得を行った。またDongら[4]は、動作と参照点の関係を、参照点に到達する、参照点から出発する、位置遷移量が等しい、などの特徴的な数パターンに大別し、動作軌跡を生成している。これらの研究はいずれも、動作の概念として参照点に対する動作軌跡を獲得することを目標としており、参照点の決定方法及び選択肢に関しては簡単な議論にとどまっている。
そのため、これらの手法では動作の目標位置の学習に十分とは言えず、表現できない動作が存在する。例えば、椅子を等間隔に並べる、という動作を考えた時、椅子一つ一つを参照点として考えても、椅子の位置関係が二つ以上の椅子の位置で決定し、かつ毎回変化しうるものであるため適切に学習することができない。このような動作を適切に学習するためには、二つ以上の物体の位置を考慮した参照点の決定方法が備わっていなければならない。本研究では、これら関連研究で用いられる手法を参考にしつつ、特に複数の物体を参照点として考慮する必要のある動作の学習を目標として、より多種の動作意図を取得する方法を提案する。        


