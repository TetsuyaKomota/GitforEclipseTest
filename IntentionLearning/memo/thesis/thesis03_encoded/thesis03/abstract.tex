近年のロボット技術の発展により、人間の生活環境で活躍する汎用ロボットの実現に向けた様々な研究が行われている。人間の生活環境で活躍する汎用ロボットには、人間とのインタラクションを通じて動作の方法や内容を学習する能力が必要不可欠である。
人間とのインタラクションを通じて学習を行う手法の研究として模倣学習があるが、人間の動作を模倣するだけではその動作を理解したとは言えず、
特定の初期状態から人間がどのように動作したかという明示的な情報から、人間がその動作を行う際に環境中のどの情報を意識しているかという暗黙的な情報を推測し、動作における観点を人間とロボット間で共有する必要がある。そのような背景から、本研究では教示者の動作から着目している観点を推測することで教示者の動作意図を把握し、意図を汲み取った動作を再現できるような、気がきくロボットの実現を目指した。

杉浦らは動詞の持つ動作概念を人とロボット間で共有することを目標にした研究を行った。その中でピックアンドプレースの教示に関して、各オブジェクトの位置を参照点とし、複数の座標系に変換した動作軌跡を生成する隠れマルコフモデルを学習することで参照点を考慮した動作概念の獲得を行った。Dongらは、動作と参照点の関係を特徴的な数パターンに大別し、動作軌跡を生成している。

本研究ではこれら関連研究の手法では実現できなかった、二つの物体の間に動かすなどの複数の物体間の相対位置を考慮した動作を学習する機構を持たせ、より多種の動作を学習できる。。。

提案手法を用いて学習させた動作を異なる初期環境で再現、識別させる実験を行うことで、複数の物体間の相対位置を考慮した動作を学習できていることを示した。

残された課題としてはチョメチョメが挙げられる.
残された課題としてはチョメチョメが挙げられる.
