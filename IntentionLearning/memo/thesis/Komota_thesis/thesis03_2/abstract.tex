人間の生活環境で活動する汎用ロボットの実現のためには,ロボットが人間とのインタラクションを通じて動作を学習する能力を持つことが望ましく,その実現を目指した研究の1つとして模倣学習がある.しかし,模倣学習の多くの先行研究では,教示動作の目標位置,すなわちその動作が目標とする最終状態について教示者とロボット間で既知で共有されているという前提に立っている.そこで本研究では,環境中の一つ以上の参照点との位置関係に応じて目標位置が決定されるような物体移動動作を教示し,着目している参照点及び動作の目標位置と参照点との位置関係を推定し,再現する手法を提案する.

初めに,環境中の参照点それぞれに学習モデルとしてガウスモデルを生成する.この際環境中の各物体の位置を参照点に含めると同時に任意の物体間の重心位置も参照点として考慮することで複数の物体の位置関係を考慮した動作の学習を可能にしている.
学習時には教示された物体移動動作を各参照点を原点とした相対位置に変換し,個々の座標系に応じた線形変換を行ったうえで各モデルを更新する.
再現時には学習されたモデルから最尤推定によって教示者が意図していたと推定される参照点を選択してベクトルを生成し,参照点と座標系に応じた逆線形変換を行うことで動作の目標位置を推定する.

また,学習した確率モデルからの生起確率を用いることで,例示動作が既学習動作のいずれであるかを識別することも可能になる.

提案手法の有効性を検証するため,提案手法を用いて学習させた動作を学習時と異なる初期環境で再現,識別させる実験を行った.結果として,複数の参照点間の相対的な位置関係を考慮した動作を学習できていることが示された.

今後の課題としては,複数の段階を持つより高度な動作の学習手法や,目標位置や被動作物体の決定時に位置情報以外の特徴量を考慮できるより一般的な学習手法の考察が挙げられる.