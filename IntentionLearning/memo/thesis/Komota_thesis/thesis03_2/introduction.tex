%\chapter{Introduction}
\chapter{序論}

\section{背景}

近年のロボット技術の発展により,人間の生活環境で活躍する汎用ロボットの実現に向けた様々な研究が行われている.汎用ロボットとは,工場などで作業を行うようなあらかじめプログラムされた動作を繰り返すものではなく,環境との相互作用によって適切な行動を学習,選択して実行できる能力を持ったロボットのことを指す.
特に,人間がロボットに動作を見せて学習させるなど人間とのインタラクションを通じて動作の方法や内容を適応的に学習する能力を持たせることで,より直感的に人間がロボットを操作することが可能になる.また,環境や動作指示者に応じて異なる動作を適切に行えるような,気が利くロボットの実現にも望ましい.

人間とのインタラクションを通じて学習を行う手法の研究の1つとして模倣学習が挙げられる.模倣学習とはロボットに対し,タスクや動作をあらかじめプログラムするのではなく,ロボットの腕を直接操作したり,人間の動作をカメラなどを通じて視覚的に与えたりといった方法で動作を教示し,その挙動を模倣することでタスクや動作を学習する手法を研究する分野であり,現在では様々な研究が行われている\cite{imitation1}-\cite{imitation4}.
中岡\cite{nakaoka}は,教示した人の舞踊動作から四肢の長さや関節角などを考慮した模倣学習を行い,同時に安定した姿勢を保ち直立を維持した状態でロボットに舞踊動作を模倣させる研究を行った.このようにロボットに対し,教示動作と類似した動作を単純に再現させる手法としての模倣学習はある程度成功している.一方,特定のタスクを達成させることを目的とした模倣学習も数多く研究されている.
Schaal\cite{schaal}は,バネマスダンパ系の運動方程式で表される運動モデルのパラメータを教示動作から学習することで,ドラムを演奏するなどの周期運動や,ラケットでボールを打つなどの到達運動の再現を行っている.このように,目標位置が既知で教示者とロボットの間で共通に認識されている,すなわち動作の目的は既知であるという前提で,その目標位置までの動作軌跡などを扱う研究は多く存在する.

しかし,動作を学習する過程においては,むしろ動作の目的の獲得方法こそ重要であると考えられる.
例えばレストランで給仕するロボットに,コーヒーを客の前に運ぶという動作を教示して再現させることを考える.動作教示を行った時点でキッチンに置かれているコーヒーの入ったコップを,その時点でテーブルの一角に座る客の前に動かす動作を見せた場合,単純に教示動作からコップの最終位置を把握したとしても,それはそのテーブルの一角という場所に動かすことが重要なのか,その座っている客の前に動かすことが重要なのかが理解できていなければ,客の座る位置やテーブルの位置などの環境が変わった場合に教示者の意図に正しく沿った動作が再現できない.このような問題を解決するためには,教示された動作の目的,すなわち教示者の動作意図をロボットが理解しなければならない.
また,動作教示の煩わしさを軽減し,人とロボットのインタラクションを円滑にするためには,そのような動作の目標位置などの教示者の内包する動作意図に関する情報は明示的に与えるより,教示動作からロボットが推測し,教示者は教示動作を見せるだけでよくなることが望ましい.そのためには,教示者の動作意図はロボット自身が教示動作から推定する機構を持つ必要がある.

このような背景から,本研究では教示者の動作から動作意図を把握し,意図を汲み取った動作を再現できるような気が利くロボットの実現に向けた研究として,一つ以上の参照点との相対的な位置関係に応じて目標位置が決定する物体移動動作の教示から,教示者の着目している参照点とそれに対する位置決定の方法を推定する手法を提案する.

\section{関連研究}

\subsection{学習モデルの設定方法}

Calinon\cite{calinon}は,与えられた教示動作から,動作の目標位置や軌跡を決定する基準点となる参照点と,参照点を基準とした変位という2つのパラメータを推定することで
,未知の環境においても教示動作に内包された動作意図をくみ取った動作再現が可能であることを示している.参照点とは,例えばコーヒーを客の前に運ぶという動作におけるコップを届ける対象となる客のことで,変位とは参照点である客の位置に対して前方という位置関係のことを指す.
\cite{calinon}では参照点と変位の推定のための学習モデルの設定方法という観点から,既存研究を以下の3つに大別できるとしている.

	\begin{enumerate}
		\item 教示動作と同数のモデルを生成し,組み合わせることで動作を再現する
		\item 参照点の候補と同数のモデルを生成し,組み合わせることで動作を再現する
		\item 1つの学習モデルを用い,パラメータを学習することで動作を再現する
	\end{enumerate}
	
Matsubaraら\cite{matsubara}は障害物をまたいで物体を移動する動作において,障害物の高さごとに変化する軌跡を持つ教示動作から教示動作ごとのモデルを学習し,統合することで未学習の高さを持つ障害物においても適切に動作を再現する手法を提案している.

杉浦ら\cite{sugiura}は動詞の持つ動作概念を人とロボット間で共有することを目標にした研究を行った.その中で物体移動動作の教示に関して,各オブジェクトの位置を参照点とし,
ロボットの視覚空間に対して平行な座標系と,動作開始点方向に軸を設定した座標系を変位の候補として,動作軌跡を生成する隠れマルコフモデル\cite{hmm}のパラメータと参照点,変位を最尤推定によって学習し,参照点を考慮した動作概念の獲得を行っている.またDongら\cite{dong}は,動作と参照点の関係である変位を,参照点に到達する,参照点から出発する,位置遷移量が等しい,などの特徴的な数パターンに大別し,動作軌跡を生成している.

\subsection{関連研究の課題点}\label{kadai}

これらの研究は参照点と変位の推定方法及び動作軌跡の獲得を目標としており,参照点や変位の候補自体の設定方法に関しては十分な議論がなされていない.
とりわけ,これらの手法で考慮されているように参照点を各物体の位置に限定した設定方法では動作の目標位置の学習に十分とは言えず,獲得できない動作が存在すると考えられる.特にこれら従来手法では学習できない動作として,複数の参照点間の相対的な位置関係を考慮した動作が挙げられる.
例えばレストランで給仕するロボットに,テーブルの横に椅子を等間隔に並べるという動作を学習させることを考える.教示動作からテーブルを参照点とするだけでは,等間隔にするという動作意図を判断できない.椅子同士が等間隔であるということを理解するためには,2つ以上の椅子の位置関係によって決定する椅子の間隔を把握する必要があり,そのためには2つ以上の椅子の位置を同時に参照点として考慮する必要がある.

本研究では,これら関連研究で用いられる手法を参考にしつつ,特に複数の物体を参照点として考慮する必要のある動作の学習を可能とし,より多種の動作意図を持つ物体移動動作を学習する方法を提案する.        

\section{本論文の構成}

本論文は次のように構成されている.第2章では従来手法の概要と課題について述べ、第3章では提案手法について従来手法と対比する形で述べる.第4章では実験とその結果について述べる.第5章では提案手法の考察と改善点,今後の課題について述べる.	
