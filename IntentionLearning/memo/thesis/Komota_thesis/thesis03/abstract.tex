人間の生活環境で活動する汎用ロボットの実現には,人間とのインタラクションを通じて動作を学習する能力が望ましく,その実現を目指した研究として模倣学習が存在する.しかし,模倣学習
の研究で扱われているものの多くの先行研究では,教示動作の目標位置即ちその動作が目標とする最終状態について教示者とロボット間で既知で共有されているという前提に立っている.そこで本研究では,環境中の一つ以上の参照点に応じて目標位置が決定されるような物体移動動作を教示し,その動作がどの参照点に対してどのように決定されるかを表す観点を推定し,再現する手法を提案する.

初めに,環境中の参照点に対応した観点それぞれに学習モデルとしてガウスモデルを生成する.この際環境中の各物体の位置を参照点に含めると同時に任意の物体間の重心位置も参照点として考慮することで複数の物体の位置関係を考慮した動作の学習を可能にしている.
学習時には教示された物体移動動作を各参照点を原点とした相対位置に変換し,観点に応じた線形変換を行ったうえで各確率モデルを更新する.
再現時には学習された確率モデルのうち,最もばらつきの小さい観点を教示者が意図していた観点だと推定してベクトルを生成し,観点に応じた逆線形変換を行ったうえで参照点の座標を加えて動作の目標位置を推定する.

また,学習した確率モデルからの生起確率を用いることで,例示動作が既学習動作のいずれであるかを識別することも可能になる.

有効性を検証するため,提案手法を用いて学習させた動作を学習時と異なる初期環境で再現,識別させる実験を行った.結果として,複数の参照点間の相対位置を考慮した動作を学習できていることが示された.

今後の課題としては,複数の段階を持つより高度な動作の学習手法や,目標位置や被動作物体の決定時に位置情報以外の特徴量を考慮できるより一般的な学習手法の考察が挙げられる.