%\chapter{Main Results}
\chapter{実験}

\section{実験環境、前提}

実験環境は2次元の有限な擬似連続空間とし、被動作対象であるトラジェクタと、参照点となりうる少量の物体が存在する。コンピュータにとって、空間の範囲、トラジェクタや物体の数、位置、観点の種類は既知で教示者と認識を共有しており、各動作における観点については未知であるとし、その観点を教示動作から獲得し、動作の再現と識別を行うことを目標とする。実験は全てシミュレータ上で行った。表\ref{table:taskname}に、実験に使用する動作の動作名と、その動作におけるトラジェクタの目標位置の対をまとめた表を示す。
\begin{table}[h]
	\caption{動作名とトラジェクタ目標位置の対応表}
	\label{table:taskname}
  	\begin{tabular}{|l|l|} \hline
    	動作名 & トラジェクタ目標位置\\ \hline
   	赤を中央に移動する & 
	$
    	\left( x_{center} , y_{center} \right)+G_{error}
    	$
    	\\
    	赤を青の右に移動する & 
	$
    	\left( x_{blue}+15 , y_{blue} \right)+G_{error}
    	$
    	\\
    	赤を橙に近づける & 
	$
    	\left( \frac{x_{red}+x_{orange}}{2} , \frac{y_{red}+y_{orange}}{2} \right)+G_{error}
    	$
    	\\
    	赤を緑から遠ざける & 
	$
    	\left( 2x_{red}-x_{green} , 2y_{red}-y_{green} \right)+G_{error}
    	$
    	\\
    	等間隔に赤、黄、青と並べる & 
	$
    	\left( 2x_{yellow}-x_{blue} , 2y_{yellow}-y_{blue} \right)+G_{error}
    	$
    	\\
    	時計回りに赤、緑、青と並べる & 
	$
	\begin{pmatrix}
        	\cos \frac{\pi}{3} & -\sin \frac{\pi}{3} \\
        	\sin \frac{\pi}{3} & \cos \frac{\pi}{3}
	\end{pmatrix}
	\begin{pmatrix}
        	x_{blue}-x_{green} \\
        	y_{blue}-y_{green}
	\end{pmatrix}
      	+
	\begin{pmatrix}
        	x_{green} \\
        	y_{green}
	\end{pmatrix}      	
	+G_{error}
    	$
    	\\ \hline
  	\end{tabular}
\end{table}

ただし、表\ref{table:taskname}における$G_{error}$とは平均0のガウス分布から生起される誤差であり、分散の大きさは各実験ごとに設定する。


\section{動作再現}

動作再現の評価方法として、ガウス誤差を含む教示動作から再現された動作の教示動作からの誤差が教示時の誤差に起因するものと観点の推定に失敗したことに起因するものがあることと、その二つを判別する方法を示した後、誤差を含む教示動作をもとに学習した際の観点推定の成功率と教示時の誤差の関係を用いる。

結果-2.結果-2.結果-2.結果-2.結果-2.結果-2.結果-2.結果-2.
結果-2.結果-2.結果-2.結果-2.結果-2.結果-2.結果-2.結果-2.
結果-2.結果-2.結果-2.結果-2.結果-2.結果-2.結果-2.結果-2.
結果-2.結果-2.結果-2.結果-2.結果-2.結果-2.結果-2.結果-2.

\section{動作識別}

動作識別の評価方法として、学習した6種類の動作のうち、例示動作がどの動作であるかを識別させ、その成功率を用いる。学習には各動作に対してガウス誤差の分散2で生成した教示動作を30回分使用し、識別する例示動作は教示動作とは異なるガウス誤差の分散10で生成した動作を各50回分使用する。識別結果は上位3動作まで求め、正しい動作が適切に識別できているか、上位3動作のうちに含まれているかで評価する。

結果-2.結果-2.結果-2.結果-2.結果-2.結果-2.結果-2.結果-2.
結果-2.結果-2.結果-2.結果-2.結果-2.結果-2.結果-2.結果-2.
結果-2.結果-2.結果-2.結果-2.結果-2.結果-2.結果-2.結果-2.
結果-2.結果-2.結果-2.結果-2.結果-2.結果-2.結果-2.結果-2.
