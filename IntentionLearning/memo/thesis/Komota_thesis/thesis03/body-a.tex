%\chapter{Preliminaries}



\chapter{提案手法}

\section{問題設定}

本研究は静的な環境下で教示される物体移動動作から、被移動物体(以下、トラジェクタ)の目標位置の決定に関与する参照点とその関係を推定する手法について提案する。
そのため本研究では、ある地点に置いてある物体をある法則に従った他の地点に移動するという、初期状態と最終状態の対を動作と定義し、動作ごとに固有の法則を観点と呼ぶ。観点は参照点と、参照点に対する位置関係を表す変位の対と定義する。

\subsection{参照点}

トラジェクタの遷移には、大別すると以下の3種類が存在する。

	\begin{enumerate}
		\item 初期状態に関わらず、トラジェクタの初期位置に対して一定の遷移を行う
		\item 初期状態に関わらず、空間上の特定の位置に遷移を行う
		\item 1つ以上の物体の位置や相対位置に応じて遷移先が変化する
	\end{enumerate}
以下に、これら3種類の違いについて図で示す。

%%%%%%%%%%%%%%%%%%%%%%%%%%%%%%%%%%%%%%%%%%%%%%%%%%%%%%%%%%%%%%%%%%%%%%%%%%%%%%%%%%%%%%%%%%%%%%%%%%%%%%%
%図が入れらんねぇ
%%%%%%%%%%%%%%%%%%%%%%%%%%%%%%%%%%%%%%%%%%%%%%%%%%%%%%%%%%%%%%%%%%%%%%%%%%%%%%%%%%%%%%%%%%%%%%%%%%%%%%%

1はトラジェクタの初期位置を、2は画面中央を参照点に含めることで、3の特殊な事例として実現できる。即ち全ての物体移動動作は、参照店である物体位置、トラジェクタの初期位置、画面中央のうちいずれかとの相対位置を考慮した目標位置を持つ。

\subsection{変位}

1。参照点を原点とし、常に一定の相対位置に遷移する
2。参照点を原点とし、トラジェクタの初期位置に応じて遷移先が変化する
3。複数の物体の位置関係に応じて遷移先が変化する
(上記3つの違いについて図で注釈)
1は例えば、物体1を物体2の右隣に動かすという動作などで、参照点である物体2に関して常に相対位置が一定である。2は、物体1を物体2に近づけるという動作などで、これは参照点となる物体2と、トラジェクタの初期位置の位置関係によって、参照点からの目標位置の相対位置が変化する。3の例として、物体1を物体2と物体3の間に動かすという動作などがあり、これは複数の物体を考慮した参照点の設定を行う必要がある。特に3を実現するために、参照点を各物体の他に、物体間の重心位置にも定める。複数の物体を考慮した目標位置の決定にはそれらの物体の重心である参照点からの相対位置を利用して学習する。

\section{従来手法}

\section{提案手法}

準備-2.準備-2.準備-2.準備-2.準備-2.準備-2.準備-2.
準備-2.準備-2.準備-2.準備-2.準備-2.準備-2.準備-2.
準備-2.準備-2.準備-2.準備-2.準備-2.準備-2.準備-2.
準備-2.準備-2.準備-2.準備-2.準備-2.準備-2.準備-2.

\section{動作学習}

\section{動作再現}

\section{動作識別}