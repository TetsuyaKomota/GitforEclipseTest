%\chapter{Introduction}
\chapter{序論}

\section{背景}

近年のロボット技術の発展により,人間の生活環境で活躍する汎用ロボットの実現に向けた様々な研究が行われている.汎用ロボットとは,工場などで作業を行うようなあらかじめプログラムされた動作を繰り返すものではなく,環境との相互作用によって適切な行動を学習,選択して実行できる能力を持ったロボットのことを指す.活動環境を事前に把握できない汎用ロボットにとって,人間とのインタラクションを通じて動作の方法や内容を適応的に学習する能力を持つことは必要不可欠である.

人間とのインタラクションを通じて学習を行う手法の研究として,人間の教示動作を観測し模倣することによって,ロボットがその動作を再現する方法を学習するというアプローチがある.しかし,単純に人間の動作を模倣するだけではその動作を意図を普遍的に理解したとは言えない.
例えばロボットに,コップを運ぶという動作を教示して再現させることを考える.動作教示を行った時点で食器棚に置かれているコップを,その時点でテーブルの一角に座る人の前に動かす動作を見せた場合,単純に教示動作からコップの最終位置を把握したとしても,それはそのテーブルの一角という場所に動かすことが重要なのか,その座っている人の前に動かすことが重要なのかが理解できていなければ,人の座る位置やテーブルの位置などの環境が変わった場合に教示者の意図に正しく沿った動作が再現できない.このような問題を解決するためには,特定の初期状態から人間がどのように動作したかという明示的な情報から,人間がその動作を行う際に環境中のどの情報を意識しているかという暗黙的な情報を推測し,動作における観点を人間とロボット間で共有する必要がある.

このような背景から,本研究では教示者の動作から動作意図を把握し,意図を汲み取った動作を再現できるようなロボットの実現に向けた研究として,一つ以上の参照点との相対位置に応じて目標位置が決定する物体移動動作の教示から,教示者の着目している参照点とそれに対する位置決定の方法を推定する手法を提案する.

\section{関連研究}

人間とロボットのインタラクションの分野において,
人間の教示動作によるロボットの模倣学習に関する様々な研究が行われている\cite{imitation1}-\cite{imitation4}.模倣学習とはロボットに対し,タスクや動作をあらかじめプログラムするのではなく,ロボットの腕を直接操作したり,人間の動作をカメラなどを通じて視覚的に与えたりといった方法で動作を教示し,その挙動を模倣することでタスクや動作を学習する手法を研究する分野である.その際に教示者とロボット間のキネマティクスなどの制約の違いを克服する機構を持たせることで,ロボットへの動作教示をより自然,適切に行う手法に関する研究が行われている.
中岡\cite{nakaoka}は,教示した人の舞踊動作から四肢の長さや関節角などを考慮した模倣学習を行い,同時に安定した姿勢を保ち直立を維持した状態でロボットに舞踊動作を模倣させる研究を行った.このようにロボットに対し,教示動作と類似した動作を単純に再現させる手法としての模倣学習はある程度成功している.一方,特定のタスクを達成させることを目的とした模倣学習も数多く研究されている.
Schaal\cite{schaal}は,バネマスダンパ系の運動方程式で表される運動モデルのパラメータを教示動作から学習することで,ドラムを演奏するなどの周期運動や,ラケットでボールを打つなどの到達運動の再現を行っている.このように,目標位置が既知で教示者とロボットの間で共通に認識されているという前提で,その目標位置までどのように動作するかを扱う研究は多く存在する.

しかし,動作教示の煩わしさを軽減し,人とロボットのインタラクションを円滑にするためには,特に動作の目標位置などの教示者の内包する意図に関する情報は明示的に与えるより,教示動作からロボットが推測し,教示者は教示動作を見せるだけでよくなることが望ましい.そのためには,目標位置はロボット自身が教示動作から推定する機構を持たなければならないが,目標位置の決定方法は動作によって異なる.例えば物体移動動作において,ある物体を右に動かすという動作はその物体の位置が最終位置の決定にかかわるが,一方の物体を他方の物体に近づけるという動作は2つの物体間の位置関係が最終位置の決定に必要になる.これには,目標位置決定の基準となる参照点と,その参照点に対してどのような位置,状態かを表す変位の2つを推定する必要がある.
この問題に対して,杉浦ら\cite{sugiura}は動詞の持つ動作概念を人とロボット間で共有することを目標にした研究を行った.その中で物体移動動作の教示に関して,各オブジェクトの位置を参照点とし,複数の座標系に変換した動作軌跡を生成する隠れマルコフモデルを学習して最尤推定によって参照点と変位を決定し,参照点を考慮した動作概念の獲得を行った.またDongら\cite{dong}は,動作と参照点の関係である変位を,参照点に到達する,参照点から出発する,位置遷移量が等しい,などの特徴的な数パターンに大別し,動作軌跡を生成している.

しかしながらこれらの研究はいずれも,動作の概念として参照点に対する動作軌跡を獲得することを目標としており,参照点の決定方法及び選択肢に関しては十分な議論がなされていない.
とりわけ,これらの手法で考慮されている参照点は動作の目標位置の学習に十分とは言えず,獲得できない動作が存在する.特にこれら従来手法では学習できない動作として,複数の参照点間の位置関係を考慮した動作が挙げられる.例えば椅子を等間隔に並べるという動作,すなわち椅子の位置関係が2つ以上の椅子の位置で決定し,かつ位置間隔は毎回変化する動作を考える.このような動作を学習する場合,1つの椅子の位置のみを参照点とするだけでは適切に学習することができないと考えられる.このような動作を適切に学習するためには,2つ以上の物体の位置を考慮した参照点の決定方法が備わっていなければならない.本研究では,これら関連研究で用いられる手法を参考にしつつ,特に複数の物体を参照点として考慮する必要のある動作の学習を目標として,より多種の動作意図を取得する方法を提案する.        

\section{本論文の構成}

本論文は次のように構成されている.第2章では従来手法の概要と課題について述べ、第3章では提案手法について従来手法と対比する形で述べる.第4章では実験とその結果について述べる.第5章では提案手法の考察と改善点,今後の課題について述べる.	
