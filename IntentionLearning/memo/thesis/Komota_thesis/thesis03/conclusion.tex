%\chapter{Conclusion}
\chapter{結論}

\section{結びと課題}

本研究では,物体移動動作において,教示者の動作を考えうる観点の候補ごとの変換を施した上でベクトルの生起確率モデルとして学習することで,教示者の動作意図を把握した目標位置の推定を行うことを目標としていた.その上で参照点の候補に参照点間の重心位置を含めることで,複数の参照点間の位置関係を考慮した物体移動動作の学習を可能にした.提案手法の課題として,全ての重心位置を参照点とすることで考慮する観点の数が物体数に対して指数関数的に増加してしまう点が挙げられる.また,今回扱った各物体の位置情報以外に,形や角度,色などの一般的な特徴量を考慮した観点推定が可能になれば,より高度な動作意図を理解することができると考えられるが,その場合本研究で扱った単純な空間座標としての変換よりも一般的で高次な座標系を設定する必要がある.それらの問題を解決する方法として,ディープラーニングの手法を用いて教示データから自動的に必要な参照点や座標系を導出することが可能になれば,より広範囲で一般的な学習手法になると考えられる.
将来的には,今回の方法で獲得した最終位置をもとに,動作軌跡の認識,推定と再生成を行いたい.
また,気がきくロボットの実現に向けて,動作教示の利便性を高めるために,多量に与えられた教示動作のうち学習に適した教示のみを選択して学習することで学習効率を上げる方法や,教示中に誤った動作を認識させてしまった場合,誤教示と判断できる動作を自動的に排除して学習を行う方法についての考察も行いたい.

