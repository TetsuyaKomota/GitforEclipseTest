%\chapter{Conclusion}
\chapter{結論}

\section{結びと課題}

本研究では、物体移動動作において、教示者の動作を考えうる観点の候補ごとの変換を施した上でベクトルの生起確率モデルとして学習することで、教示者の動作意図を把握した目標位置の推定を行うことを目標としていた。その上で参照点の候補に参照点間の重心位置を含めることで、複数の参照点間の位置関係を考慮した物体移動動作の学習を可能にした。将来的には、今回の方法で獲得した最終位置をもとに、動作軌跡の認識、推定と再生成を行いたい。
また、今回扱った各物体の位置情報以外に、形や角度、色などの一般的な特徴量を考慮した観点推定が可能になれば、より高度な動作意図を理解することができると考えられる。
また、気がきくロボットの実現に向けて、動作教示の利便性を高めるために、多量に与えられた教示動作のうち学習に適した教示のみを選択して学習することで学習効率を上げる方法や、教示中に誤った動作を認識させてしまった場合、誤教示と判断できる動作を自動的に排除して学習を行う方法についての考察も行いたい。

